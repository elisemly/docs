\documentclass[norsk,a4paper]{article}
\usepackage[norsk]{babel}
\usepackage[utf8]{inputenc}
\usepackage{parskip,textcomp,fullpage,fancyhdr}
\usepackage[T1]{fontenc}
\pagestyle{fancy}
\fancyhf{}
\renewcommand{\headrulewidth}{0pt}

% Redefines sections and subsections in the format §1-1
\renewcommand{\thesection}{§\arabic{section}}
\renewcommand{\thesubsection}{\thesection-\arabic{subsection}}

\usepackage{libertine}

\title{Samarbeidsavtale}

%%%%%%%%%%%%%%%%%%%%%%%%%%%%%%%%%%%%%%%%%%%%%%%%%%%%%%%%%%%%%%%%%%%%%%%%%%%%%%%
% Metadata
%%%%%%%%%%%%%%%%%%%%%%%%%%%%%%%%%%%%%%%%%%%%%%%%%%%%%%%%%%%%%%%%%%%%%%%%%%%%%%%

\begin{document}

\maketitle
\thispagestyle{fancy}

\section{Avtalens omfang}
Dette er en samarbeidsavtale mellom DNB ASA og studentforeningen Ifi-dagen ved
Institutt for Informatikk, som inneholder beskrivelse av rettigheter, plikter,
tjenester og pris for samarbeidet. Avtalen inngås mellom Ifi-dagen
(org.nr.\ 987 042 583) og DNB ASA (org.nr. 984 851 006) --- heretter
hovedsamarbeidspartneren.

\subsection{Avtalens tidsomfang}
Avtalen gjelder fra 10.\ april til 31.\ oktober.

\subsection{Kontaktpersoner mellom partene}
Ifi-dagen stiller med sin bedriftsansvarlig --- Egwene Tegelaár --- som
kontaktperson.

Hovedsamarbeidspartneren stiller med Fredrik Svendsen som sin kontaktperson.

\subsubsection{Bytte av kontaktperson}
Ved eventuell bytte av kontaktperson forplikter parten som bytter å sende
skriftlig varsel om bytte, minimum 1 uke før byttet blir gjennomført.

\section{dagen@ifi}
\subsection{Arrangementer\label{sec:arrangementer}}
Ifi-dagen skal i løpet av perioden holde to arrangementer: ettermiddagen@ifi
og dagen@ifi. For begge disse arrangementene skal hovedsamarbeidspartneren ha
ordinær standplass. Spesifikt for dagen@ifi kan hovedsamarbeidspartneren fritt
innrede et barområde kalt for ``loungen'' som del av arrangementet som foregår
på kvelden.

Videre kan Ifi-dagen stille med lokaler for eventuelle andre arrangementer
hovedsamarbeidspartneren ønsker å holde, så framt det er ved Institutt for
Informatikk.

\subsection{Promotering}
Ifi-dagen skal tydelig promotere for hovedsamarbeidspartneren på sine
nettsider, samt ved siden av logoen sin på alt av promoteringsmateriale i
relasjon til arrangementet dagen@ifi, inkludert det av goder som blir gitt til
frivillige funksjonærer som jobber under dagen@ifi.

\section{DNB}
\subsection{Forpliktelser}
Hovedsamarbeidspartneren forplikter seg til å holde masterkickoff onsdag 22.
august 2018\ i samarbeid med Ifi-dagen. Her stiller hovedsamarbeidspartneren
seg ansvarlig for de økonomiske omkostningene som oppstår. Økonomiske aspekter
ved dette står i~\ref{sec:okonomi}.

\subsection{Økonomiske forpliktelser\label{sec:okonomi}}
Hovedsamarbeidspartneren forplikter seg til å utbetale et beløp på NOK
100\ 000,-- eksl.\ mva.\ for samarbeid. Denne kostnaden kommer i tillegg til
eventuelle andre kostnader.

For masterkickoff og eventuelle andre arrangementer i regi av
hovedsamarbeidspartner forplikter hovedsamarbeidspartneren seg til å betale for
mat, lokale og drikke tilknyttet arrangementet, med unntak av under de forhold
gitt i~\ref{sec:arrangementer} andre ledd. Nøyaktige detaljer rundt mengde og
pris vil utarbeides av partene via deres kontaktpersoner.

\section{Vilkår}
Partene forplikter seg til å gjennomføre arbeidet i samsvar med gjeldende lover
og forskrifter samt de regler og retningslinjer som er relevante for
gjennomføringen av samarbeidet, herunder etiske regler og retningslinjer samt
anerkjente kvalitetsstandarder og normer.

\subsection{Force Majeure}
Dersom avtalens gjennomføring helt eller delvis hindres, eller i vesentlig grad
vanskeliggjøres av forhold som ligger utenfor partenes kontroll, suspenderes
partenes plikter i den utstrekning forholdet er relevant, og for så lang tid
som forholdet varer. Slike forhold inkluderer, men er ikke begrenset til,
streik, lockout, og ethvert forhold som etter norsk rett vil bli bedømt som
force majeure.

Den som rammes av force majeure må varsle den annen part om dette innen rimelig
tid, Det skal også gis varsel når force mejeure-situasjonen opphører. Unnlatt
varsel kan føre til erstatningsansvar eller at den forpliktede mister retten
til å påberope seg force majeure.

\subsection{Konflikter}
Uenighet om gjennomføring, tjenesterelevanse eller innhold i avtalen, skal
forsøkes løst ved forhandlinger mellom partene.

Hvis partene ikke kommer til enighet, skal konflikten avgjøres ved volgift.

\subsection{Taushetsplikt}
Partene plikter å bevare taushet om konfidensiell informasjon om den annen part
uansett formidlingsform --- herunder detaljer rundt tjenester som leveres, og
øvrige forhold som avtalepartene etter vanlig vurdering bør forstås er av
betydning å hemmeliggjøre av hensyn til den annen parts virksomhet. Ett unntak
for dette er gjennomgang av økonomiske forhold under generalforsamling.

Informasjon som er alminnelig tilgjengelig regnes ikke som konfidensiell
informasjon. Konfidensiell informasjon skal kun anvendes som en del av egen
aktivitet og kun i den utstrekning det er nødvendig. Partene kan overføre
informasjon til utenforstående i den utstrekning dette er nødvendig for
gjennomføring av avtalen, forutsatt at slik annen mottaker av informasjon
pålegges en plikt til konfidensialitet slik det fremgår av dette punkt.
Bestemmelsene i dette punkt er ikke til hinder for at partene kan utnytte
erfaring og kompetanse som opparbeides i forbindelse med gjennomføringen av
avtalen.

\section*{}

Avtalen er underskrevet i to eksemplarer, der partene innehar hvert sitt
eksemplar.\\[2\bigskipamount]

\textbf{Dato og sted:}\\[5\bigskipamount]

\noindent
\begin{tabular}{@{}p{2.5in}p{2.5in}@{}}
    \hrulefill{} & \hrulefill{} \\
    Karl Hole Totland & Ivy Lotarev \\
    Leder i Ifi-dagen & Seksjonsleder ved IT Culture Transformation under IT
                        Transformation\\[5\bigskipamount]
    \hrulefill{} & \\
    Thor Marius Kokvik Høgås \\
    Økonomiansvarlig i Ifi-dagen
\end{tabular}

\end{document}
